% -----
% ARQUIVO: capitulo-02.tex
% VERSÃO: 1.1
% DATA: Janeiro de 2016
%
% CAPÍTULO DE INTRODUÇÃO DA PROPOSTA
%
% NÃO MEXA NAS SEÇÕES, SOMENTE EDITE O CONTEÚDO.
% -----

\chapter{Introdução} \label{chap:Intro}
% #TXT_INTRODUCAO
O avanço tecnológico que possibilitou a redução de tamanho de processadores, aliado ao aumento de capacidade de processamento abriu caminho para uma proliferação de equipamentos integrados com sensores, atuadores, RFIDs (Radio-Frequency IDentification), smartphones e dispositivos sem fio a serem utilizados para os mais diversos propósitos. Havia uma tendência natural que esses dispositivos se conectassem à internet a fim de compartilhar os dados coletados com nós geograficamente dispersos e permitir que usuários pudessem enviar comandos para esses objetos de remotamente.

Ao longo do tempo, mesmo que não tenhamos dado conta, essas "coisas" começaram a fazer parte de nossas vidas alterando significativamente a forma como os indivíduos se relacionam.

%Lembrando que serão utilizados os termos dispositivos, objetos e coisas para representar qualquer equipamento conectado à internet.

Para representar essa extensão da internet que rompeu os limites das redes de computadores, participando de forma pervasiva no dia a dia das pessoas, foi cunhado o termo Internet das Coisas.

De acordo com \cite{000-000} a Internet das coisas nasceu com o propósito ousado de conectar todas as coisas do mundo na internet. O que está causando um grande impacto no dia-a-dia e no comportamento das pessoas, \cite{000-003}, uma vez que está se inserindo em diversas áreas como por exemplo transporte, ambientes domésticos, indústria, logística, saúde e entretenimento. A cada dia um novo sistema ou aplicação é incluído nesse ecossistema, de forma que hoje, a grande maioria de nossas atividades diárias estão sendo controladas ou monitoradas por um dispositivo conectado à internet.

A indústria e a academia têm trabalhado na construções de ambientes compostos por sistemas inteligentes em diversas áreas como casas inteligentes, transporte, logística e saúde \cite{000-000}.

%Outra linha de pesquisa tem aprofundado a questão da arquitetura dos dispositivos como em \cite{000-006} no qual é proposta uma análise abrangente do estado da arte da Internet das coisas abordando uma proposta de desenho de arquitetura de três camadas: (i) Tecnológica, (ii) Acesso (iii) Internet, (iv) \textit{Middleware} e (v) Aplicação. Já em \cite{000-003} o autor aprofunda o assunto com foco no desenvolvimento para dispositos de Internet das Coisas, detalhando os quesitos técnicos da arquitetura.
%
%Porém, uma das mais importantes questões que tem surgido é como converter os dados gerados ou capturados pela Internet das Coisas em conhecimento \cite{000-000}. Então diversas técnicas de KDD (\textit{Knowledge Discovery in Databases}) e Mineração de dados têm sido aplicadas para analisar os dados com o objetivo de melhorar os serviços providos pelos ambientes formados por esses dispositivos.

\section{Contextualização} \label{sec:context}
% #TXT_CONTEXTUALIZACAO

A quantidade de dados geradas pelos dispositivos gera grandes oportunidades de extrair informações valiosas com a utilização de técnicas de KDD (\textit{Knowledge Discovery in Databases}) e Mineração de dados, mas por outro lado traz grandes desafios no gerenciamento dos dados devido ao volume, heterogeneidade, dispersão e questões temporais, dos dados gerados pelas coisas.

De acordo com \cite{000-000} a quantidade de dados gerados em um ano já excede a 1 zetabyes ($10^{21}$ bytes). As ferramentas de análise de dados disponíveis atualmente não são poderosas o suficiente para lidar com uma quantidade tão grande de dados, isso sem contar com as questões de armazenamento e capacidade de transmissão de todo esse volume de informação entre as coisas e os nós centrais de processamento. Curiosamente o gargalo no tratamento dos dados, nesse contexto, aos poucos é transferido dos sensores para o processamento, comunicação e armazenamento.

Os dados na Internet das Coisas podem ser classificados em dados sobre as coisas, como estado, localização, etc e dados gerados pelas coisas. Esses últimos contém dados representativos das interações entre: (i) coisas, (ii) humanos e coisas e (iii) entre humanos \cite{000-018}. 

Assim, ao utilizar técnicas de detecção de padrões é possível identificar o comportamento das coisas e dos humanos cujos dados são coletados pelos dispositivos, que por sua vez possibilitam identificar desvios comportamentais dos objetos e pessoas monitorados. 

\section{Trabalhos Relacionados} \label{sec:trabRelac}

Os trabalhos relacionados foram divididos em três grupos: \begin{itemize}
    \item \textbf{Mineração de Dados na Internet das Coisas}: Apresentam o estado da arte da Mineração de dados na Internet das Coisas. Apresentam os desafios e os problemas abertos.
    \item \textbf{Mineração de Dados Distribuída}: Trabalhos com técnicas de mineração distribuída, não necessariamente no contexto da Internet das Coisas. \linebreak
    Apresentarão bases para análise e projeto dos algoritmos e técnicas que serão desenvolvidos  para o atingimento do objetivo.
    \item \textbf{Detecção de Desvios}: Trabalhos relacionados com identificação de padrões com geração de modelos e detecção de desvios de comportamento no monitoramento de atividades.
    \end{itemize}

\subsection{Mineração de Dados na Internet das Coisas}

\begin{enumerate}

    \item \cite{000-000}: Um Survey abrangente sobre mineração de dados no contexto da Internet das Coisas. Discorre sobre as diversas técnicas de mineração de dados aplicadas ao contexto da Internet das Coisas relacionando-as com seus usos: infraestrutura e serviços. Apresenta os principais problemas abertos, divididos nos seguintes grupos: \begin{itemize}
        \item Infraestrutura
        \item Dados
        \item Algoritmos
        \item Segurança e Privacidade
    \end{itemize}
    Aborda o fato dos principais algoritmos de mineração de dados serem centralizados, o que pode, em pouco tempo inviabilizar suas utilizações em Internet das Coisas devido à sobrecarga causada nas redes de dispositivos na transferência de grande quantidade de dados.

    \item \cite{000-014}: São abordados os desafios na mineração de redes de sensores sem fio -- WSN (\textit{Wireless Sensor Networks}. Tem como foco analisar o trade-off entre: \begin{itemize}
        \item Mineração centralizada com alto consumo de rede para transferência de dados
        \item Mineração nos sensores com limitações: \begin{itemize}
            \item Capacidade de processamento
            \item Consumo de Energia
            \item Tempo de vida das baterias
        \end{itemize}
    \end{itemize}
 
    \item \cite {003-000}: A partir da preocupação com o grande volume de dados, o trabalho propõe quatro diferentes modelos de mineração de dados para Internet das coisas: \begin{itemize}
        \item Modelo multicamada
        \item Modelo de Mineração Distribuída
        \item Mineração de Dados baseada em Grid
        \item Modelo baseado numa perspectiva multitecnológica.
    \end{itemize}

    Após a descrição dos modelos, são apresentados alguns dos problemas abertos e uma discussão sobre a utilização de mineração centralizada versus descentralizada. 
    
\end{enumerate}

\subsection{Mineração de Dados Distribuída}

\begin{enumerate}

    \item \cite{001-000}: Nesse trabalho é apresentado o GDCluster, um algoritmo genérico de clusterização descentralizada. 
    Tem como proposta que cada nó faça uma sumarização de seus dados e compartilhe com seus vizinhos. O resultado final é atingido analisando os dados internos com os dados sumarizados recebidos dos demais nós. Isso é feito de forma iterativa até que se chegue no nó central.
    O algoritmo considera o dinamismo dos dados, atribuindo uma idade a cada item do \textit{dataset}.
    
    % Colocar figura? 
    
    \item \cite{017-000}: Apresenta o $D^{2}CA$ (\textit{Distributed Dynamic Clustering Algorithm}), um algoritmo que explora a clusterização de forma distribuída, maximizando o paralelismo e minimizando a comunicação. Tem como rotina básica: \begin{itemize}
        \item Modelos locais gerados por clusterização K-means
        \item Cálculo dos contornos (usados como representantes dos clusters)
        \item Troca de contornos entre nos vizinhos
        \item Fusão dos contornos em grupos
        \item Escolha de um líder entre os nós
        \item Repetição do processo até atingir o nó central.
    \end{itemize}
    

\end{enumerate}

\subsection{Detecção de Desvios}

\begin{enumerate}
    \item \cite{022-000}: Propõe uma nova formulação na detecção de \textit{outliers} com base na distância, utilizando um algoritmo baseado em partições
    \item \cite{023-000}: Utiliza o algoritmo LOF (\textit{Local Outlier Factor}) aplicado em um grid com o objetivo de reduzir o tempo de computação na descoberta de desvios de comporamento.
\end{enumerate}

\section{Problema} \label{sec:problema}

O grande volume gerado sobre e pelas coisas necessitam de análise de dados não só pelas especificidades de cada aplicação (negócio ou infraestrutura) mas também no sentido de compreender o modo como a Internet das Coisas está se inserindo na vida das pessoas assim como os impactos que causará na sociedade.

Considerando o crescente volume de dados neste cenário, a tradicional reunião dos dados coletados de forma distribuída visando análises centralizadas tende a ser inviável \cite{000-000}.

Aliado ao exposto, muitas aplicações de Internet das Coisas estão voltadas à segurança das pessoas, como os trabalhos apresentados por \cite{000-109}, \cite{000-112}, \cite{000-113} e \cite{000-114} que abordam métodos de monitoramento de atividades diárias de pessoas idosas em residência inteligente e/ou identificação de desvios de comportamento. Nessas aplicações busca-se conhecer o padrão de comportamento de cada pessoa ou grupo de pessoas de forma a identificar padrões suspeitos (desvios que divirjam da normalidade) a fim de desencadear ações de auxílio direcionadas.

\section{Hipótese} \label{sec:hipotese}
% PARTE DE HIPÓTESES DE PESQUSA - TANTAS QUANTO NECESSÁRIO
\begin{hipo}
O uso de técnicas de mineração de dados distribuída no cenário da Internet das Coisas pode contribuir para reduzir o tráfego de informações necessárias às análises de desvios de padrões de comportamento sem diminuir a qualidade dos padrões que seriam identificados por meio de técnicas de mineração de dados centralizada.\end{hipo}

\section{Objetivo} \label{sec:obj}
Demonstrar que o uso de técnicas de mineração de dados distribuída no cenáario da Internet das Coisas pode contribuir para reduzir o tráfego de informações necessárias às análises de desvios de padrões de comportamento sem diminuir a qualidade dos padrões que seriam identificados por meio de técnicas de mineração de dados centralizada.

\section{Contribuições} \label{sec:contrib}
Todo mundo vê os porris que eu tomo, mas ninguém vê os tombis que eu levo! Interessantiss quisso pudia ce receita de bolis, mais bolis eu num gostis. Quem num gosta di mé, boa gente num é. Atirei o pau no gatis, per gatis num morreus.

Copo furadis é disculpa de bebadis, arcu quam euismod magna. Cevadis im ampola pa arma uma pindureta. Per aumento de cachacis, eu reclamis. Suco de cevadiss deixa as pessoas mais interessantiss.

Mé faiz elementum girarzis, nisi eros vermeio. Posuere libero varius. Nullam a nisl ut ante blandit hendrerit. Aenean sit amet nisi. Si num tem leite então bota uma pinga aí cumpadi! in elementis mé pra quem é amistosis quis leo. 

As contribui\c{c}\~{o}es esperadas para este trabalho s\~{a}o:

\begin{enumerate}[label=(\roman*)]
\item Neque porro quisquam est, qui dolorem ipsum quia dolor sit amet, consectetur, adipisci velit, sed quia non numquam eius modi tempora incidunt ut labore et dolore magnam aliquam quaerat voluptatem.

\item At vero eos et accusamus et iusto odio dignissimos ducimus qui blanditiis praesentium voluptatum deleniti atque corrupti quos dolores et quas molestias excepturi sint occaecati cupiditate non provident, similique sunt in culpa qui officia deserunt mollitia animi, id est laborum et dolorum fuga.
\end{enumerate}


