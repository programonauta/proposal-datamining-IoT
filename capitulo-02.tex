% -----
% ARQUIVO: capitulo-02.tex
% VERSÃO: 1.1
% DATA: Janeiro de 2016
%
% CAPÍTULO DE INTRODUÇÃO DA PROPOSTA
%
% NÃO MEXA NAS SEÇÕES, SOMENTE EDITE O CONTEÚDO.
% -----

\chapter{Introdução} \label{chap:Intro}
% #TXT_INTRODUCAO
O avanço tecnológico que possibilitou a redução de tamanho de processadores, aliado ao aumento de capacidade de processamento abriu caminho para uma proliferação de equipamentos integrados com sensores, atuadores, RFIDs (Radio-Frequency IDentification), smartphones e dispositivos sem fio a serem utilizados para os mais diversos propósitos. Havia uma tendência natural que esses dispositivos se conectassem à internet a fim de compartilhar os dados coletados com nós geograficamente dispersos e permitir que usuários pudessem enviar comandos para esses objetos de remotamente.

Ao longo do tempo, mesmo que não tenhamos dado conta, essas "coisas" começaram a fazer parte de nossas vidas alterando significativamente a forma como os indivíduos se relacionam.

%Lembrando que serão utilizados os termos dispositivos, objetos e coisas para representar qualquer equipamento conectado à internet.

Para representar essa extensão da internet que rompeu os limites das redes de computadores, participando de forma pervasiva no dia a dia das pessoas, foi cunhado o termo Internet das Coisas.

De acordo com \cite{000-000} a Internet das coisas nasceu com o propósito ousado de conectar todas as coisas do mundo na internet. O que está causando um grande impacto no dia-a-dia e no comportamento das pessoas, \cite{000-003}, uma vez que está se inserindo em diversas áreas como por exemplo transporte, ambientes domésticos, indústria, logística, saúde e entretenimento. A cada dia um novo sistema ou aplicação é incluído nesse ecossistema, de forma que hoje, a grande maioria de nossas atividades diárias estão sendo controladas ou monitoradas por um dispositivo conectado à internet.

A indústria e a academia têm trabalhado na construções de ambientes compostos por sistemas inteligentes em diversas áreas como casas inteligentes, transporte, logística e saúde \cite{000-000}.

%Outra linha de pesquisa tem aprofundado a questão da arquitetura dos dispositivos como em \cite{000-006} no qual é proposta uma análise abrangente do estado da arte da Internet das coisas abordando uma proposta de desenho de arquitetura de três camadas: (i) Tecnológica, (ii) Acesso (iii) Internet, (iv) \textit{Middleware} e (v) Aplicação. Já em \cite{000-003} o autor aprofunda o assunto com foco no desenvolvimento para dispositos de Internet das Coisas, detalhando os quesitos técnicos da arquitetura.
%
%Porém, uma das mais importantes questões que tem surgido é como converter os dados gerados ou capturados pela Internet das Coisas em conhecimento \cite{000-000}. Então diversas técnicas de KDD (\textit{Knowledge Discovery in Databases}) e Mineração de dados têm sido aplicadas para analisar os dados com o objetivo de melhorar os serviços providos pelos ambientes formados por esses dispositivos.

\section{Contextualização} \label{sec:context}
% #TXT_CONTEXTUALIZACAO

A quantidade de dados geradas pelos dispositivos gera grandes oportunidades de extrair informações valiosas com a utilização de técnicas de KDD (\textit{Knowledge Discovery in Databases}) e Mineração de dados, mas por outro lado traz grandes desafios no gerenciamento dos dados devido ao volume, heterogeneidade, dispersão e questões temporais, dos dados gerados pelas coisas.

De acordo com \cite{000-000} a quantidade de dados gerados em um ano já excede a 1 zetabyes ou $10^{21}$ bytes. As ferramentas de análise de dados disponíveis atualmente não são poderosas o suficiente para lidar com uma quantidade tão grande de dados, isso sem contar com as questões de armazenamento e capacidade de transmissão de todo esse volume de informação entre as coisas e os nós centrais de processamento. Curiosamente o gargalo no tratamento dos dados, nesse contexto, aos poucos é transferido dos sensores para o processamento, comunicação e armazenamento.

Os dados na Internet das Coisas podem ser classificados em dados sobre as coisas, como estado, localização, etc e dados gerados pelas coisas. Esses últimos contém dados representativos das interações entre: (i) coisas, (ii) humanos e coisas e (iii) entre humanos \cite{000-018}. 

Assim, ao utilizar técnicas de detecção de padrões é possível identificar o comportamento das coisas e dos humanos cujos dados são coletados pelos dispositivos, que por sua vez possibilitam identificar desvios comportamentais dos objetos e pessoas monitorados. 

\section{Trabalhos Relacionados} \label{sec:trabRelac}
Cacilds vidis litro abertis. Todo mundo vê os porris que eu tomo, mas ninguém vê os tombis que eu levo! Em pé sem cair, deitado sem dormir, sentado sem cochilar e fazendo pose. Quem num gosta di mé, boa gente num é. in elementis mé pra quem é amistosis quis leo.

Nullam volutpat risus nec leo commodo, ut interdum diam laoreet. Sed non consequat odio. Nec orci ornare consequat. Praesent lacinia ultrices consectetur. Sed non ipsum felis. Paisis, filhis, espiritis santis. Cevadis im ampola pa arma uma pindureta.

Mais vale um bebadis conhecidiss, que um alcoolatra anonimiss. Vehicula non. Ut sed ex eros. Vivamus sit amet nibh non tellus tristique interdum. Pra lá , depois divoltis porris, paradis. Manduma pindureta quium dia nois paga.

\section{Problema} \label{sec:problema}
Sapien in monti palavris qui num significa nadis i pareci latim. Diuretics paradis num copo é motivis de denguis. Atirei o pau no gatis, per gatis num morreus. Si num tem leite então bota uma pinga aí cumpadi!

A ordem dos tratores não altera o pão duris Suco de cevadiss, é um leite divinis, qui tem lupuliz, matis, aguis e fermentis. Delegadis gente finis, bibendum egestas augue arcu ut est. Quem manda na minha terra sou Euzis!

Posuere libero varius. Nullam a nisl ut ante blandit hendrerit. Aenean sit amet nisi. Viva Forevis aptent taciti sociosqu ad litora torquent Vehicula non. Ut sed ex eros. Vivamus sit amet nibh non tellus tristique interdum. undefined 

\section{Hipótese} \label{sec:hipotese}
Praesent malesuada urna nisi, quis volutpat erat hendrerit non. Nam vulputate dapibus. Delegadis gente finis, bibendum egestas augue arcu ut est. Em pé sem cair, deitado sem dormir, sentado sem cochilar e fazendo pose. Suco de cevadiss, é um leite divinis, qui tem lupuliz, matis, aguis e fermentis.
% PARTE DE HIPÓTESES DE PESQUSA - TANTAS QUANTO NECESSÁRIO
\begin{hipo}
Lorem ipsum dolor sit amet, consectetur adipiscing elit, sed do eiusmod tempor incididunt ut labore et dolore magna aliqua?
\end{hipo}



Nullam volutpat risus nec leo commodo, ut interdum diam laoreet. Sed non consequat odio. Per aumento de cachacis, eu reclamis. Quem num gosta di mé, boa gente num é. Cevadis im ampola pa arma uma pindureta.

Posuere libero varius. Nullam a nisl ut ante blandit hendrerit. Aenean sit amet nisi. Ta deprimidis, eu conheço uma cachacis que pode alegrar sua vidis.” Todo mundo vê os porris que eu tomo, mas ninguém vê os tombis que eu levo! Nec orci ornare consequat. Praesent lacinia ultrices consectetur. Sed non ipsum felis. 

\section{Objetivo} \label{sec:obj}
Quem manda na minha terra sou Euzis! A ordem dos tratores não altera o pão duris Admodum accumsan disputationi eu sit. Vide electram sadipscing et per. Diuretics paradis num copo é motivis de denguis.

Per aumento de cachacis, eu reclamis. Sapien in monti palavris qui num significa nadis i pareci latim. Interessantiss quisso pudia ce receita de bolis, mais bolis eu num gostis. Atirei o pau no gatis, per gatis num morreus.

Si u mundo tá muito paradis? Toma um mé que o mundo vai girarzis! Paisis, filhis, espiritis santis. Suco de cevadiss deixa as pessoas mais interessantiss. Quem num gosta di mé, boa gente num é. 

\section{Contribuições} \label{sec:contrib}
Todo mundo vê os porris que eu tomo, mas ninguém vê os tombis que eu levo! Interessantiss quisso pudia ce receita de bolis, mais bolis eu num gostis. Quem num gosta di mé, boa gente num é. Atirei o pau no gatis, per gatis num morreus.

Copo furadis é disculpa de bebadis, arcu quam euismod magna. Cevadis im ampola pa arma uma pindureta. Per aumento de cachacis, eu reclamis. Suco de cevadiss deixa as pessoas mais interessantiss.

Mé faiz elementum girarzis, nisi eros vermeio. Posuere libero varius. Nullam a nisl ut ante blandit hendrerit. Aenean sit amet nisi. Si num tem leite então bota uma pinga aí cumpadi! in elementis mé pra quem é amistosis quis leo. 

As contribui\c{c}\~{o}es esperadas para este trabalho s\~{a}o:

\begin{enumerate}[label=(\roman*)]
\item Neque porro quisquam est, qui dolorem ipsum quia dolor sit amet, consectetur, adipisci velit, sed quia non numquam eius modi tempora incidunt ut labore et dolore magnam aliquam quaerat voluptatem.

\item At vero eos et accusamus et iusto odio dignissimos ducimus qui blanditiis praesentium voluptatum deleniti atque corrupti quos dolores et quas molestias excepturi sint occaecati cupiditate non provident, similique sunt in culpa qui officia deserunt mollitia animi, id est laborum et dolorum fuga.
\end{enumerate}


